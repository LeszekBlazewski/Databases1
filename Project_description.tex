\documentclass{article}
\usepackage[utf8]{inputenc}
\usepackage[T1]{fontenc}
\usepackage{natbib}


\title{Bazy danych 1}

\author{Leszek Błażewski, 241264 \\ \\Karol Noga, 241259}

\date{Semestr letni 2018/2019}

\begin{document}
\maketitle
\clearpage

\section{Główne założenia projektu}
Głównym celem projektu jest stworzenie oraz poprawne zamodelowanie przykładowej bazy danych pozwalającej na przechowywanie danych, które wykorzystywane są przez aplikacje trzecie. Wraz z bazą danych utworzona zostanie aplikacja desktopowa/webowa pozwalająca na interakcję z danymi zawartymi w bazie. Aplikacja umożliwia przeprowadzanie podstawowych operacji na danych typu: usuń, wyszukaj, zmień, dodaj. Ostatnim elementem załączonym w projekcie będzie generator danych losowych napisany w postaci skryptu, którego egzekucja pozwoli na szybkie wypełnienie bazy danymi. Do projektu zostanie również stworzona dokumentacja pozwalająca na dokładne zapoznanie się z strukturą bazy danych oraz zawierająca opis interfejsu użytkownika.

\section{Wybrane tematy}
\subsection{System obsługi spółdzielni mieszkaniowej}

\subsubsection{Opis}
Projekt ma na celu usprawnienie działania spółdzielni mieszkaniowej. Baza danych służy jako podstawowe źródło przechowywania informacji dotyczących danych związanych z zarządzaniem spółdzielnią mieszkaniową.

\subsubsection{Przykładowe encje}
W bazie przechowywane będą dane pozwalające na kompleksowe zarządzanie całą infrastrukturą spółdzielni.
\vspace{1mm}

Przykładowe encje składowane w bazie:
 \begin{itemize}
    \item Mieszkaniec
    \item Mieszkanie
    \item Blok
    \item Ulica
    \item Dozorca
    \item Osiedle
\end{itemize}

\subsubsection{Udostępniane widoki}
Grupą docelową jest administrator oraz sztab dozorców nadzorujących dane osiedle. W związku z czym aplikacja oferować będzie dwa widoki odpowiednie dla każdego z podmiotów.

\pagebreak{}

Widoki zawarte w aplikacji:
\begin{enumerate}
    \item Widok administratora
    \item Widok dozorcy
\end{enumerate}
1. Widok administratora pozwala na zażądanie całą bazą. Administrator w dowolnej chwili może zmieniać dane w wszystkich tabelach. Jego widok pozwala na walidację oraz swobodne przeglądanie wszystkich danych.\linebreak
2. Widok dozorcy pozwala na przeglądanie danych mieszkańców oraz ich miejsca zamieszkania oraz modyfikację jedynie swoich danych w bazie.

\subsubsection{Przykładowe operacje podmiotów}
Przykładowe operacje administratora spółdzielni:
\begin{itemize}
    \item Sprawdzenie jakie osoby zamieszkują dane mieszkanie.
    \item Sprawdzenie ile wynosi czynsz za dane lokum.
    \item Przydzielanie dozorców do konkretnych dzielnic.
\end{itemize}
Przykładowe operacje dozorcy:
\begin{itemize}
    \item Zmiana swoich danych w bazie.
    \item Przeglądanie godzin pracy innych dozorców.
    \item Przeglądanie danych właściciela danego mieszkania.
\end{itemize}

\subsection{System zarządzania pływalnią}

\subsubsection{Opis}
Głównym celem projektu jest wsparcie systemu zarządzania pływalnią. Implementacja bazy danych pozwoli na łatwe przechowywanie oraz zarządzanie danymi klientów oraz pracowników pływalni. 

\subsubsection{Przykładowe encje}
W bazie przechowywane będą dane pozwalające na kompleksowe zarządzanie całą infrastrukturą pływalni.
\vspace{1mm}

Przykładowe encje składowane w bazie:
 \begin{itemize}
    \item Klient
    \item Ratownik
    \item Basen
    \item Rezerwacja
\end{itemize}

\subsubsection{Udostępniane widoki}
Aplikacja oferować będzie dwa widoki, które pozwolą podmiotom na poprawną oraz bezpieczną interakcję z danymi zawartymi w bazie.
\vspace{1mm}

Widoki zawarte w aplikacji:
\begin{enumerate}
    \item Widok administratora
    \item Widok klienta
\end{enumerate}
1. Widok administratora pozwala na zażądanie całą bazą. Administrator w dowolnej chwili może zmieniać dane w wszystkich tabelach. Jego widok pozwala na walidację oraz swobodne przeglądanie wszystkich danych.\linebreak
2. Widok klienta pozwala na przeglądanie możliwych terminów rezerwacji oraz na wszystkie operacje związane z ich zarządzaniem.

\subsubsection{Przykładowe operacje podmiotów}
Przykładowe operacje administratora:
\begin{itemize}
    \item Sprawdzenie kto dokonał danej rezerwacji.
    \item Sprawdzenie jaki ratownik przydzielony jest do danego basenu.
    \item Zarządzanie danymi ratowników oraz klientów.
\end{itemize}
Przykładowe operacje klienta:
\begin{itemize}
    \item Kupno biletu na daną godzinę.
    \item Rezerwacja danego basenu przez wyznaczony termin.
    \item Usunięcie wcześniej założonej rezerwacji.
\end{itemize}

\subsection{Wykorzystane technologie}
\begin{itemize}
    \item DMBS: ORACLE
    \item Wersja bazy danych: Oracle Database 11g Express Edition
    \item Konteneryzacja: docker
    \item Aplikacja desktopowa: C\#[Winforms]/Java[Swing]
    \item Skrypt generujący dane: Python
\end{itemize}
\end{document}
